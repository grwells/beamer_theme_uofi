\documentclass[aspectratio=169]{beamer}

%% PACKAGES
%\usepackage{tikz}
\usepackage[utf8]{inputenc}
\usepackage[T1]{fontenc}
\usepackage[style=ieee]{biblatex}
\usepackage{url}
\usepackage{minted}
\usepackage{pifont}  % special symbols
\usepackage{animate}
%\usepackage{multimedia}
\usepackage{amsmath} % matrix typesetting
\usetikzlibrary{matrix}
\usepackage{hyperref} % links

%% BIBLIOGRAPHY
\setbeamertemplate{bibliography item}{\insertbiblabel}
\addbibresource{../bibliography/Lecture-10.bib}

%% BEAMER OUTPUT FORMAT
%\setbeameroption{hide notes} % Only slides
%\setbeameroption{show only notes} % Only notes
\setbeameroption{show notes on second screen=right} % Both

%% THEME SETTINGS
\usetheme{default}
\usecolortheme{crane}
\usetheme{uofi}

%% TITLE SLIDE OPTIONS
\title{Lecture \#10}
\subtitle{Thresholding}
\date{\textit{revised \today}}
\author{Garrett Wells}


\begin{document}

%% TITLE SLIDE 
\begin{frame}
    \titlepage
\end{frame}

%% TOC
\section{Table of Contents}
\begin{frame}[allowframebreaks]{Table of Contents}
    \tableofcontents
\end{frame}

\section{Review}
\begin{frame}{Review}
\end{frame}

\begin{frame}{Thresholding}
    \framesubtitle{\textit{Introduction\dlots}}
    \centering
    \begin{tikzpicture}
        \draw[black, very thick, ->] (yaxis) (-5, 0) -- node[left](-5, 2.5){pixel value} (-5, 5);
        \draw[black, very thick, ->] (xaxis) (-5, 0) -- (5, 0);

        \draw[blue, very thick] (p3) (-3, 0) -- (-3, 0.8) node[above]{$0.8$};
        \draw[magenta, very thick] (p1) (-3.5, 0) -- (-3.5, 2) node[above]{$2$};
        \draw[green, very thick] (p2) (-4, 0) -- (-4, 4) node[above]{$4$};
    \end{tikzpicture}
\end{frame}


%% MINTED EXAMPLE
\subsection{Simple}
\begin{frame}[fragile]{Simple Thresholding}
    \begin{block}{Simple Thresholding, Generalized}     
        Applies the same threshold value, $threshold$ to every pixel in the image. 
        \vspace{0.5cm}
        \begin{minted}{python}
if pixel_val >= threshold:
    pixel_val = upper_bound
else:
    pixel_val = lower_bound
        \end{minted}
    \end{block}

\end{frame}

\begin{frame}[fragile]{Simple Thresholding}
  \begin{minted}{python}
cv.threshold(
      src,    # source image
      thresh, # threshold value
      maxval, # maximum output value
      type,   # what type of thresholding to use
    )
  \end{minted}
\end{frame}

%% FOOTNOTE/PIFONT EXAMPLE
\subsubsection{Inverted Binary}
\begin{frame}{Inverted Binary Threshold}
    \begin{itemize}
        \item values over threshold \ding{222} 0
        \item values below threshold \ding{222} max value
    \end{itemize}
    \begin{figure}
      \centering
      \includegraphics[width=0.7\textwidth]{../include/pictures/binary_thresh_wave.png}
      \caption{Binary Threshold\footnotemark}
    \end{figure}

    \footnotetext{OpenCV Image Processing \cite{OpenCVMiscellaneousImage}}
\end{frame}

%% BIBLIOGRAPHY SLIDES
\begin{frame}[allowframebreaks]{Bibliography}
\printbibliography
\end{frame}

\end{document}
